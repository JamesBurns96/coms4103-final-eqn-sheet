\documentclass[landscape,a4paper]{article}
\usepackage[utf8]{inputenc}
\usepackage{amsmath}
\usepackage{amssymb}
\usepackage{gensymb}
\usepackage{multicol}
\usepackage{setspace}
\usepackage{tabularx}
\usepackage[margin=1cm]{geometry}
\usepackage{graphicx}
\usepackage{subcaption}
\usepackage{float}
\usepackage{siunitx}
\usepackage{mathrsfs}
\graphicspath{ {images/} }
\setlength{\parskip}{0cm}
\setlength{\parindent}{1em}
\usepackage[compact]{titlesec}
\titlespacing{\section}{0pt}{2ex}{1ex}
\titlespacing{\subsection}{0pt}{1ex}{0ex}
\titlespacing{\subsubsection}{0pt}{0.5ex}{0ex}
\usepackage{paralist}
\usepackage{listings}

\begin{document}

\begin{multicols}{3}
	\section{Constants}
	\begin{align}
		\mu_0 &= 4\pi \times 10^{-7}\\
		\epsilon_0 &= 8.854 \times 10^{-12}\\
		Y_0 &= \frac{1}{377 \Omega}
	\end{align}
	
	\section{Equations - Joel}
	\begin{align}
		Y &= Y_0 n\\
		H &= YE\\
		B &= \mu_0 H
	\end{align}
	\textbf{Snell's Law: }
	\begin{align}
		n_1 \sin(\theta_1) &= n_2 \sin(\theta_2)
	\end{align}
	\textbf{Critical Angle:}
	\begin{align}
		\theta_c &= \sin^{-1}\left( \frac{n_2}{n_1} \right)
	\end{align}
	\textbf{Brewster's Angle:}
	\begin{align}
		\theta_B &=  \arctan \left( \frac{n_2}{n_1} \right)
	\end{align}
	\textbf{Reflection coeff.}
	\begin{align}
		\Gamma &= \frac{E_r}{E_i} = \frac{n_1 - n_2}{n_1 + n_2}\\
		\text{Energy: } R  &=  |\Gamma|^2
	\end{align}
	\textbf{Transmission coeff.}
	\begin{align}
		\tau = \frac{E_t}{E_i} &= \frac{2n_1}{n_1 + n_2}\\
		T = \frac{n_2}{n_1} |\tau|^2 &= \frac{4 n_1 n_2}{(n_1 + n_2)^2}
	\end{align}
	\textbf{TIR - S-polarised:}
	\begin{align}
		\delta &= \frac{1}{\alpha}\\
		%\alpha &=  k_0 \sqrt{n_1^2 \sin^2 \theta_i - n_2^2}\\
		\alpha &= \frac{2\pi n_2}{\lambda_0} \sqrt{\left( \frac{n_1}{n_2} \right)^2 \sin^2\theta_i - 1}\\
		%TODO: Which Alpha? Seems the second is just the first but with k_0 given an actual value and n_2 taken out of the \sqrt component
		\Delta z &= 2\delta \tan \theta_i
	\end{align}
	\textbf{Diffraction:}\\
	\begin{align}
		r &= \sqrt{(x'-x)^2 + z^2}\\
		R &= \sqrt{(x')^2 + z^2}
	\end{align}
	\hspace*{3mm}\textbf{Far field:}
	\begin{align}
		g(x') &= \frac{e^{-jkR}}{R} \int f(x) e^\frac{jkx'x}{R} dx\\
		g(\omega) &= \int f(t) e^{j\omega t} dt
	\end{align}
	\hspace{3mm}\textbf{Rayleigh Criterion:}
	\begin{align}
		\Delta x &= \frac{1.22 \lambda f}{D}\\
		\Delta \theta &= \frac{1.22 \lambda}{D}\\
		f = \text{focal le}&\text{ngth, } D = \text{aperture diameter} \nonumber
	\end{align}
	\hspace{3mm}\textbf{Airy disk:}
	\begin{align}
		\sin \theta &= 1.22 \frac{\lambda}{D}\\
		\tan \theta &= \frac{r}{f}
	\end{align}
	\hspace{3mm}\textbf{Bragg diffraction grating:}
	\begin{align}
		d(\sin \theta_m - \sin \theta_i) = m \lambda, \hspace{2mm}m \epsilon \mathbb{Z}
	\end{align}
	\textbf{Coherence:}
	\begin{align}
		l_c &= c \tau_c\\
		\tau_c &=  \frac{1}{\Delta \upsilon}, \hspace{2mm} \Delta \upsilon = \text{spectral width}
	\end{align}
	\textbf{2D Waveguide:}\\
	\hspace{3mm}\textbf{Path length (reflections):}
	\begin{align}
		L_p &= \frac{2p}{\sin \theta_z}
	\end{align}
	\hspace{3mm}\textbf{Ray 1/2 period:}
	\begin{align}
		z_p &= \frac{2p}{\tan \theta_z}
	\end{align}
	\hspace{3mm}\textbf{Reflections per unit length:}
	\begin{align}
		N &= \frac{1}{z_p}
	\end{align}
	\hspace{3mm}\textbf{Transit time:}
	\begin{align}
		t &= \frac{z n_{core}}{c \cos \theta_z}
	\end{align}
	\hspace{3mm}\textbf{Ray Invariant:}
	\begin{align}
		\beta &= n_{core} \cos \theta_z = n_{cladding} \cos \theta_t\
	\end{align}
	\hspace{3mm}\textbf{Graded Index Waveguide:}
	\begin{align}
		\theta_c (x=0) &= \cos^{-1} \left( \frac{n_{cl}}{n_{co}} \right)\\
		L_p &= \int_{-x_{tp}}^{x_{tp}} \frac{n(x) dx}{\sqrt{n^2 (x) - \beta^2}}\\
		z_p &= \beta \int_{-x_{tp}}^{x_{tp}} \frac{dx}{\sqrt{n^2(x) - \beta^2}}\\
		t &= \frac{1}{c} \int n(x) ds\\
		n(r) &= 
		\begin{cases}
			n_{co} \sqrt{1 - 2\Delta \left( \frac{x}{\rho} \right)^\alpha} & |x| < \rho\\
			n_{cl} & |x| > \rho
		\end{cases}\\
		\Delta &= \frac{n_{co}^2 - n_{cl}^2}{2n_{co}^2}
	\end{align}
	\textbf{Single/multi mode fibre:}
	\begin{align}
		V &= \frac{2\pi a}{\lambda} \sqrt{n_1^2 - n_2^2}\\
		M &= \frac{V^2}{2}\\
		M_{single} &< 2.405
	\end{align}
	\hspace{3mm}\textbf{Numerical aperture:}
	\begin{align}
	NA &= \sqrt{n_{co}^2 - n_{cl}^2} \approxeq n_{co} \sqrt{2\Delta}\\
	\theta_a &= \sin^{-1} (NA)
	\end{align}
	\textbf{Coupling:}
	%TODO: fix coupling notes so I include counter-directional, and make the distinction between that and codirectional
	\begin{align}
		2\delta &= \Delta \beta = \beta_a - \beta_b\\
		\frac{dA}{dz} &= -j\kappa B e^{-j2\delta z}\\
		\frac{dB}{dz} &= -j\kappa A e^{j2\delta z}
	\end{align}
	\hspace{3mm}\textbf{Max transfer:}
	\begin{align}
		z\sqrt{\kappa^2 + \delta^2} &= \frac{\pi}{2} + m\pi
	\end{align}
	\hspace{3mm}\textbf{Complete transfer length:}
	\begin{align}
		L_c &= \frac{\pi}{2\sqrt{\kappa^2 + \delta^2}}
	\end{align}
	\hspace{3mm}\textbf{Distributed Bragg Reflector:}
	\begin{align}
		\lambda_B &= 2 n_{eff} \Lambda
	\end{align}
	\textbf{Polarisation:}\newline
	\hspace{3mm}\textbf{Wave Plates:}
	\begin{align}
		\Phi &= \frac{2\pi}{\lambda} |n_e - n_o| L
	\end{align}
	\hspace{3mm}\textbf{Stokes:}
	\begin{align}
		\begin{bmatrix}
			S_1 \\ S_2 \\ S_3 \\ S_4
		\end{bmatrix}
		&=
		\begin{bmatrix}
			I_{H,+45,R} + I_{V,-45,L} \\
			I_H - I_V \\
			I_{+45} - I_{-45} \\
			I_R - I_L
		\end{bmatrix} \\
		p &= \frac{\sqrt{S_1^2 + S_2^2 + S_3^2}}{S_0}
	\end{align}
	\hspace{3mm}\textbf{Jones:}
	\begin{align}
		J &=
		\begin{bmatrix}
			A_{x\epsilon\mathbb{C}} \\ A_{y\epsilon\mathbb{C}}
		\end{bmatrix}
	\end{align}
	\textbf{Resonators:}
	\begin{align}
		FSR &= \frac{c}{2nL}\\
		FWHM &= \frac{FSR}{\pi} \frac{1 - \sqrt{R_1 R_2}}{(R_1 R_2)^\frac{1}{4}}\\
		Q &= \frac{2\pi n L}{\lambda} \frac{(R_1 R_2)^\frac{1}{4}}{1 - \sqrt{R_1 R_2}}\\
		\mathscr{F} &= \frac{FSR}{FWHM}\\
		&= \frac{\pi\times (R_1 R_2)^\frac{1}{4}}{1 - \sqrt{R_1 R_2}}
	\end{align}
	\hspace{3mm}\textbf{Fabry-Perot Resonator:}
	\begin{align}
		E_{tn} &= E_0 \tau_1 \tau_2 e^{-j\beta L} (\Gamma_1 \Gamma_2 e^{-j 2 \beta L})^n\\
		E_t &= \Sigma_{i=0}^\infty E_{ti}\\
		\tau_{cavity} &= \frac{E_t}{E_0} = \frac{\tau_1 \tau_2 e^{-j \beta L}}{1 - \Gamma_1 \Gamma_2 e^{-j2 \beta L}}\\
		\upsilon_q &= \frac{qc}{2nL},\hspace{2mm} q \epsilon \mathbb{Z^+}
	\end{align}
	\hspace{3mm}\textbf{Photon lifetime/density:}
	\begin{align}
		%\frac{\Delta N_p}{\Delta t} &= \frac{N_P (t = \tau_{RT}) - N_p (t = 0)}{\tau_{RT}}\\
		%&= -\frac{(1 - R_1 R_2) N_p}{\tau_{RT}}
		\tau_{RT} &= \frac{2nL}{c}\\
		\tau_p &= \frac{\tau_{RT}}{1 - R_1 R_2}\\
		N_p (t) &= N_p (0) e^{-\frac{t}{\tau_p}}\\
		I = (h\nu)\upsilon_g N_p &= \frac{1}{2} Y_0 n |E|^2\\
		I(z) &= I_0 e^{-\alpha z}\\
		\alpha &=  \langle\alpha_i\rangle + \alpha_m\\
		\alpha_m &= \frac{1}{2L} \ln \frac{1}{R_1 R_2}
	\end{align}
	
	\section{Equations - Tina}
	\textbf{Semiconductors:}\\
	\hspace{3mm}\textbf{Intrinsic Concentration:}
	\begin{align}
		n_i &= \sqrt{N_c N_v} e^{-\frac{E_g}{2 k_B T}}
	\end{align}
	\hspace{3mm}\textbf{Conductivity/Resistivity:}
	\begin{align}
		\sigma &= e n_i (\mu_e + \mu_h)\\
		\rho &= \frac{1}{\sigma}
	\end{align}
	\hspace{3mm}\textbf{Electrons/holes:}
	\begin{align}
		n &= N_c e^{-\frac{E_C - E_F}{k_B T}}\\
		p &= N_v e^{-\frac{E_F - E_V}{k_B T}}
	\end{align}
	\textbf{LEDs:}\\
	\hspace{3mm}\textbf{Wavelength:}
	\begin{align}
		\lambda &= \frac{c}{v} = \frac{hc}{E_{ph}}\\
		\Delta \lambda &= \lambda^2 \frac{3 k_B T}{hc}
	\end{align}
	\hspace{3mm}\textbf{Escape Cone:}
	\begin{align}
		\frac{P_{escape}}{P_{source}} &= \frac{\phi_c^2}{4} \approx \frac{1}{4n_s^2}\quad \text{for } n_2=1
	\end{align}
	\textbf{Photodiode:}\\
	\hspace{3mm}\textbf{Efficiency:}
	\begin{align}
		\eta_{QE} &= \frac{hcR}{e\lambda}\\
		\eta &= \frac{I_{ph}}{P_o} \frac{h\nu}{e}
	\end{align}
	\hspace{3mm}\textbf{Received Power:}
	\begin{align}
		P_o &= \frac{I_{ph}}{R}\\
		I_o &= \frac{P_o}{A}
	\end{align}
	\hspace{3mm}\textbf{Photodetector Noise:}
	\begin{align}
		\sigma_S^2 &= \sigma_Q^2 + \sigma_{DB}^2\\
		\sigma_Q^2 &= 2eI_P BM^2 F(M)\\
		\langle i_{DB}^2 \rangle = \sigma_{DB}^2 &= 2eI_D BM^2 F(M)\\
		\langle i_T^2 \rangle = \sigma_T^2 &= \langle i_{Load}^2 \rangle + \langle i_{Amp}^2 \rangle = \frac{4kT}{R_L}BF_N
	\end{align}
	\textbf{Uncategorised:}\\
	\hspace{3mm}\textbf{Relaxation resonance:}
	\begin{align}
		\omega_R &= \sqrt{\frac{a v_g \Gamma \eta_i}{qd} (J-J_{th})}
	\end{align}
	\hspace{3mm}\textbf{Threshold gain:}
	\begin{align}
		g_{th} &= \frac{\alpha_i + \alpha_m}{\Gamma}
	\end{align}
	\hspace{3mm}\textbf{Differential Gain:}
	\begin{align}
		a &= \frac{dg}{dN}
	\end{align}
	\hspace{3mm}\textbf{Laser Diode:}
	\begin{align}
		v_m = m\frac{c}{2nL},&\quad\delta v_m = \frac{c}{2nL}\\
		\lambda_m = \frac{2nL}{m},&\quad\delta\lambda_m = \frac{\lambda_m^2}{2nL}
	\end{align}
	\textbf{Calculus:}
	\begin{align}
		(f(g(x)))' &= f'(g(x))g'(x)
	\end{align}
\end{multicols}


\end{document}

%check:
%- blazed grating notes
%- in the graded index waveguide, the fuck is with the integrals? What parts are and aren't we integrating?
%- in graded index waveguide expression for t, did we mean ds or dx?
%
%add:
%- diffraction on a rectangular aperture
%- \overrightarrow{S_{av}} &= \frac{1}{2} Y_0 n |\overrightarrow{E}|^2 \overrightarrow{e_k}
%- , determines number of modes for waveguide, relates to \frac{\pi}{2} somehow
